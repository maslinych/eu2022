\documentclass[svgnames]{beamer}


\mode<presentation>
{
  \usetheme[titleformat=smallcaps,numbering=fraction,progressbar=frametitle]{metropolis}
  \usecolortheme[light,accent=orange]{solarized}
  %\usecolortheme[named=Goldenrod]{structure}
  % or ...

  \setbeamercovered{transparent}
  % or whatever (possibly just delete it)
}


% \usepackage{mathtext}
\usepackage[utf8]{inputenc}
\usepackage[english,russian]{babel}
\usepackage{cmap}
\hypersetup{unicode=true}
\graphicspath{{images/}{slides/images}}


\title[QTA 02] % (optional, use only with long paper titles)
{Препроцессинг}

\subtitle
{Квантитативный анализ текста} % (optional)

\author%[Author, Another] % (optional, use only with lots of authors)
{Кирилл Александрович Маслинский}
% - Use the \inst{?} command only if the authors have different
%   affiliation.

\institute%[Universities of Somewhere and Elsewhere] % (optional, but mostly needed)
{НИУ ВШЭ Санкт-Петербург}
% - Use the \inst command only if there are several affiliations.
% - Keep it simple, no one is interested in your street address.

\date%[Short Occasion] % (optional)
{14.02.2022 / 02}

\subject{natural language processing, text mining}
% This is only inserted into the PDF information catalog. Can be left
% out. 


\AtBeginSubsection[]
{
  \begin{frame}<beamer>[plain]{План}
    \tableofcontents[sectionstyle=show/hide,subsectionstyle=show/shaded/hide]
  \end{frame}
}

\newcommand{\tb}[1]{\colorbox{yellow}{#1}\space}
\newcommand{\Sp}[1]{\colorbox{green}{#1}\space}
\newcommand{\Sn}[1]{\colorbox{red}{#1}\space}


\begin{document}

\begin{frame}
  \titlepage
\end{frame}

\begin{frame}
  \frametitle{Как считать слова}
  \Large
  Чтобы оценить частотное распределение слов, нам необходимо посчитать
  количество употреблений  (\structure{tokens}) каждого слова
  (\structure{type}) в тексте. 
  
  \begin{block}{Qs:}
  \begin{itemize}
  \item Что является токеном? (Что считать, а что не считать?)
  \item Какие токены считать одним и тем же словом?
  \end{itemize}
  \end{block}
\end{frame}

\section{Токены}

\begin{frame}
\LARGE
  \frametitle{Токенизация}
\only<1>{
  \begin{block}{Сколько токенов?}
    Ой какие фотки<smile006><smile006><smile006> А разве роды в 38недель не считаются нормой?
  \end{block}
}
\only<2>{
  \begin{block}{11? (разделим по пробелам)}
    \tb{Ой} \tb{какие} \tb{фотки<smile006><smile006><smile006>} \tb{А} \tb{разве} \tb{роды} \tb{в} \tb{38недель} \tb{не} \tb{считаются} \tb{нормой?}
  \end{block}
}
\only<3>{
  \begin{block}{11? (возьмем только слова)}
    \tb{Ой} \tb{какие} \tb{фотки}<smile006><smile006><smile006> \tb{А} \tb{разве} \tb{роды} \tb{в} 38\tb{недель} \tb{не} \tb{считаются} \tb{нормой}?
  \end{block}
}
\only<4>{
  \begin{block}{13? (пунктуация тоже нужна)}
    \tb{Ой} \tb{какие} \tb{фотки} \tb{<smile006><smile006><smile006>} \tb{А}
    \tb{разве} \tb{роды} \tb{в} 38\tb{недель} \tb{не} \tb{считаются}
    \tb{нормой} \tb{?}
  \end{block}
}
\only<5>{
  \begin{block}{14? (всё-таки исправим опечатку)}
    \tb{Ой} \tb{какие} \tb{фотки} \tb{<smile006><smile006><smile006>} \tb{А}
    \tb{разве} \tb{роды} \tb{в} \tb{38} \tb{недель} \tb{не} \tb{считаются}
    \tb{нормой} \tb{?}
  \end{block}
}
\only<6>{
  \begin{block}{16? (посчитаем смайлики раздельно)}
    \tb{Ой} \tb{какие} \tb{фотки} \tb{<smile006>} \tb{<smile006>} \tb{<smile006>} \tb{А}
    \tb{разве} \tb{роды} \tb{в} \tb{38} \tb{недель} \tb{не} \tb{считаются}
    \tb{нормой} \tb{?}
  \end{block}
}
\end{frame}

\begin{frame}

  \frametitle{N-граммы}

  N последовательно стоящих друг за другом слов.

  \begin{block}{Униграммы}
    \alert<2>{Восторг} \alert<3>{внезапный} \alert<4>{ум}
    \alert<5>{пленил} .
  \end{block}

  \begin{block}{Биграммы}
  \alert<2>{Восторг} \alert<2-3>{внезапный} \alert<3-4>{ум}
  \alert<4-5>{пленил} \alert<5>{<.>}
  \end{block}

  \begin{block}{Триграммы}
  \alert<2>{<s>}  \alert<2-3>{Восторг} \alert<2-4>{внезапный}
  \alert<3-5>{ум} \alert<4-5>{пленил} \alert<5>{<.>}
  \end{block}
\end{frame}

\section{Типы}

\begin{frame}
  \frametitle{Стемминг}

  \LARGE
\only<1>{
  \begin{block}{How many types?}
    Кукушка кукушонку купила капюшон. Кукушонок в капюшоне смешон.
  \end{block}
}
\only<2>{
  \begin{block}{How many types?}
    Кукушка кукушонку купила \alert{капюшон}. Кукушонок в капюшон\alert{-е} смешон.
  \end{block}
}
\only<3>{
  \begin{block}{How many types?}
    Кукушка кукушон\alert{-ку} купила капюшон. \alert{к}укушон\alert{-ок} в капюшоне смешон.
  \end{block}
}
\only<4>{
  \begin{block}{How many types?}
    \alert{к}укуш\alert{-ка} кукуш\alert{-онку} купила капюшон. кукуш\alert{-онок} в капюшоне смешон.
  \end{block}
}
\only<5>{
  \begin{block}{How many types?}
    кукуш кукуш купи капюшон. кукуш в капюшон смеш.
  \end{block}
}

\hrulefill\normalsize
  \begin{description}
  \item[стемминг / stemming] — cut a word to its stem
  \end{description}
\end{frame}

% \begin{frame}
%   \frametitle{Stemming for Russian}
%   \begin{itemize}
%   \item Porter stemmer
%   \item Stemka
%   \end{itemize}
% \end{frame}


\begin{frame}
  \frametitle{Морфологический анализ: лемматизация}
  \LARGE
  \begin{block}{Сколько слов?}
    кукушка кукушонок купить капюшон. кукушонок в капюшон смешной.
  \end{block}

\hrulefill\normalsize
  \begin{description}
  \item[лемматизация / lemmatization] — приведение слова к начальной форме
  \end{description}
\end{frame}


\begin{frame}
  \frametitle{Омонимия языковых знаков}
  \LARGE
  \begin{block}{Одно слово или разные?}
    Косил \alert<1>{косой косой косой}.
  \end{block}
\only<1>{\normalsize
  коса=S,жен,неод=твор,ед\\
  косая=S,жен,од=(род,ед|дат,ед|твор,ед|пр,ед)\\
  косой=S,муж,од=им,ед\\
  косой=A=(им,ед,полн,муж|род,ед,полн,жен|
  дат,ед,полн,жен|вин,ед,полн,муж,неод|твор,ед,полн,жен| пр,ед,полн,жен)
}
\only<2>{
  \alert{косить}=V,несов=прош,ед,изъяв,муж,пе\\
  \alert{косой}=S,муж,од=им,ед\\
  \alert{косой}=A=твор,ед,полн,жен\\
  \alert{коса}=S,жен,неод=твор,ед
}
\end{frame}

\begin{frame}
  \frametitle{Морфологический анализ для русского}
  \begin{itemize}
  \item Mystem (леммы, части речи, грамматические формы, снятие
    омонимии), прямо в R
  \item udpipe (то же + синтаксические функции), прямо в R
  \item Stanza (то же + синтаксические функции), нейронная сеть: требуется python, медленный
  \item DeepPavlov (то же + синтаксические функции), нейронная сеть: требуется python, медленный
  \end{itemize}
\end{frame}

\begin{frame}
  \frametitle{Терминология}
  \begin{description}
  \item[корпус] — здесь: исследуемая коллекция текстов
  \item[token] — словоупотребление, минимальный сегмент текста
  \item[словоформа / wordform] — слово в тексте, измененное — падеж, время и т.п.
  \item[лексема / lexeme] — слово в словаре, совокупность всех форм
  \item[стемминг / stemming] — урезание слова до основы
  \item[лемматизация / lemmatization] — приведение слова к начальной форме
  \end{description}
\end{frame}

\section{Стоп-слова и частотные списки}

\begin{frame}
  \frametitle{Стоп-слова} Простейший способ уменьшить число
  лексических переменных — просто удалить \alert{наименее информативные} слова:

  \begin{itemize}
  \item Статический список:

без
более
бы
был
была
были
было
быть
в
вам
вас
весь
во
вот
все
всего
всех
вы
где
да
даже
для ...

  \item Динамический список:

    \begin{itemize}
    \item Слишком частотные (N самых частотных; частотность больше k)
    \item Слишком редкие (частотность меньше k)
    \item Слишком короткие (меньше M букв)
    \item По документной частоте (присутствующие более чем в k\%
      текстов или менее чем в k текстах)
    \end{itemize}
  \end{itemize}
\end{frame}

\end{document}
